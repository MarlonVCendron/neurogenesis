\setlength{\absparsep}{18pt} % ajusta o espaçamento dos parágrafos do resumo

\begin{resumo}

A neurogênese adulta no giro denteado (DG) do hipocampo está implicada em diversas funções da memória, desde a codificação e
consolidação até o esquecimento. Apesar das evidências experimentais sobre seus efeitos macro, os mecanismos computacionais pelos
quais as células granulares imaturas (iGCs) geradas pela neurogênese adulta modulam o circuito hipocampal permanecem pouco
compreendidos. Estudos sobre o tema frequentemente se concentram nos efeitos locais no DG, negligenciando o impacto em áreas
subsequentes como o Cornu Ammonis 3 (CA3), e tratam as iGCs como uma população estática, uma simplificação que não captura a
dinâmica de sua maturação. Este projeto propõe o desenvolvimento de um modelo computacional do circuito DG-CA3 baseado em dados
experimentais para investigar essas questões. Para isso, será utilizado o simulador Brian2, com neurônios Izhikevich, sinapses com
plasticidade de curto prazo e de longo prazo dependente do tempo de disparo. O trabalho analisará como as iGCs, durante sua fase
de maior excitabilidade e após sua maturação em células maduras (mGCs), influenciam não apenas a separação de padrões do DG, mas
também as funções de autoassociação e completamento de padrões do CA3. O modelo simulará a maturação temporal das iGCs e a adição
contínua de novos neurônios, além de incorporar as retroprojeções do CA3 para o DG, buscando assim uma compreensão mais
aprofundada dos mecanismos pelos quais a neurogênese modula a dinâmica do circuito hipocampal. Espera-se que o DG apresente uma
codificação mista, onde mGCs se especializam na separação de padrões e iGCs na integração. Acredita-se, também, que a alta
excitabilidade das iGCs, por um lado, prejudique o completamento de padrões no CA3 através da ativação promíscua de assembleias e,
por outro, possibilite a integração temporal de informações ao longo de sua maturação, modulando a consolidação de memórias.

\noindent\textbf{Palavras-chave}: Neurogênese adulta, hipocampo, modelo computacional, plasticidade, giro denteado, CA3.
\end{resumo}

\begin{resumo}[Abstract]
\begin{otherlanguage*}{english}
  
Adult neurogenesis in the hippocampal dentate gyrus (DG) is implicated in diverse memory functions, ranging from encoding and
consolidation to forgetting. Despite extensive experimental evidence on its macro-level effects, the computational mechanisms by
which immature granule cells (iGCs) generated through adult neurogenesis modulate the hippocampal circuit remain poorly
understood. Studies on the topic often focus on local DG effects, neglecting the impact on downstream regions such as Cornu
Ammonis 3 (CA3), and treat iGCs as a static population, a simplification that fails to capture the dynamics of their maturation.
This project proposes the development of a DG-CA3 computational model based on experimental data to investigate these questions.
To this end, we will employ the Brian2 simulator, using Izhikevich neurons and synapses endowed with both short-term plasticity
and spike-timing-dependent long-term plasticity. The study will analyze how iGCs, during their peak excitability phase and after
their maturation into mature granule cells (mGCs), influence not only pattern separation in the DG, but also the autoassociation
and pattern completion functions of CA3. The model will simulate the temporal maturation of iGCs and the continuous addition of
new neurons, as well as incorporate CA3 to DG feedback projections, thereby aiming for a deeper understanding of the mechanisms by
which neurogenesis modulates hippocampal circuit dynamics. We expect that the DG will exhibit a mixed coding scheme, with mGCs
specializing in pattern separation and iGCs in integration. It is also hypothesized that the high excitability of iGCs will, on
one hand, impair CA3 pattern completion via promiscuous assembly activation and, on the other hand, enable the temporal
integration of information across their maturation, thus modulating memory consolidation.

\noindent \textbf{Keywords}: Adult neurogenesis, hippocampus, computational model, plasticity, dentate gyrus, CA3.
\end{otherlanguage*}
\end{resumo}



