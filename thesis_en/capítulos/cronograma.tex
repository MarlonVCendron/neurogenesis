\chapter{Cronograma}

\definecolor{v}{HTML}{A4C6A4}
\definecolor{x}{HTML}{808080}
\definecolor{b}{HTML}{FFFFFF}

\newcommand{\rowlabel}[1]{\parbox[c][3.2cm][c]{4.5cm}{\centering\textbf{#1}}}
\newcommand{\firstcolcell}[1]{\parbox[c][1.6cm][c]{4.5cm}{\raggedright #1}}

\begin{table}[H]
    \centering
    \caption{Cronograma de atividades}
    \begin{tabularx}{\textwidth}{|p{4.5cm}|*{8}{>{\centering\arraybackslash}X|}}
        \hline
        \multirow{2}{*}{} & \multicolumn{8}{c|}{\textbf{Trimestre}} \\
        \cline{2-9}
        & \textbf{1} & \textbf{2} & \textbf{3} & \textbf{4} & \textbf{5} & \textbf{6} & \textbf{7} & \textbf{8} \\
        \hline
        \firstcolcell{Revisão de literatura}
        & \cellcolor{v} & \cellcolor{v} & \cellcolor{v} & \cellcolor{v} & \cellcolor{x} & \cellcolor{x} & \cellcolor{x} & \cellcolor{b} \\
        \hline
        \firstcolcell{Implementação do modelo do DG}
        & \cellcolor{v} & \cellcolor{v} & \cellcolor{v} & \cellcolor{v} & \cellcolor{b} & \cellcolor{b} & \cellcolor{b} & \cellcolor{b} \\
        \hline
        \firstcolcell{Implementação do modelo do CA3}
        & \cellcolor{b} & \cellcolor{v} & \cellcolor{v} & \cellcolor{v} & \cellcolor{x} & \cellcolor{x} & \cellcolor{b} & \cellcolor{b} \\
        \hline
        \firstcolcell{Escrita da dissertação}
        & \cellcolor{b} & \cellcolor{b} & \cellcolor{b} & \cellcolor{b} & \cellcolor{x} & \cellcolor{x} & \cellcolor{x} & \cellcolor{x} \\
        \hline
        \firstcolcell{Análise da separação de padrões}
        & \cellcolor{b} & \cellcolor{b} & \cellcolor{b} & \cellcolor{v} & \cellcolor{b} & \cellcolor{b} & \cellcolor{x} & \cellcolor{b} \\
        \hline
        \firstcolcell{Implementação da maturação temporal}
        & \cellcolor{b} & \cellcolor{b} & \cellcolor{b} & \cellcolor{b} & \cellcolor{x} & \cellcolor{x} & \cellcolor{x} & \cellcolor{b} \\
        \hline
        \firstcolcell{Análise da autoassociação e completamento de padrões}
        & \cellcolor{b} & \cellcolor{b} & \cellcolor{b} & \cellcolor{b} & \cellcolor{b} & \cellcolor{b} & \cellcolor{x} & \cellcolor{x} \\
        \hline
    \end{tabularx}
    \label{tab:cronograma}
    \vspace{1em}
    \par
    \noindent \colorbox{v}{\phantom{XX}} \hspace{1ex} Feito
    \hspace{2em}
    \colorbox{x}{\phantom{XX}} \hspace{1ex} A fazer
\end{table}

