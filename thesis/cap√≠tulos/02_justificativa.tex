\chapter{Justificativa}

Muito embora já existam diversos trabalhos que tentam definir uma função para a neurogênese, o papel desempenhado pelas células
imaturas geradas por ela no circuito que integram, bem como o que ocorre com esse circuito durante sua maturação ainda não são
entendidos~\cite{aimoneComputational2016, aimoneModeling2011, berdugo-vegaSharpening2023, faresNeurogenesis2019}.

Diversos estudos acerca desse tema têm se concentrado principalmente nos efeitos da neurogênese no próprio DG e na separação de
padrões realizada por tal~\cite{berdugo-vegaSharpening2023, kimAdult2024,wangEffect2024}, negligenciando seu impacto nas regiões
conectadas, como a área CA3. Portanto, este projeto visa preencher essa lacuna ao examinar também como as iGCs afetam as funções
de memória da área CA3 (autoassociação e completamento de padrões), e como as retroprojeções dessa área para o DG influenciam o
sistema~\cite{myersPattern2011}, visando um modelo mais holístico e biologicamente realista.

Também é comum que diversos trabalhos estudem a função de uma população de iGCs estática~\cite{aimoneComputational2016,
berdugo-vegaSharpening2023}, sendo muito raros modelos que levem em consideração a maturação dessas células através do tempo e a
substituição de mGCs~\cite{aimoneComputational2009}. Esses estudos podem explicar a função (ou falta de) dessas células imaturas
apenas enquanto imaturas, mas não são capazes de explicar como essa fase impacta essas células ao maturarem.

Portanto, este estudo se justifica pela sua abordagem inovadora, preenchendo as lacunas existentes na literatura ao
simular a maturação das iGCs e suas consequências no CA3.


