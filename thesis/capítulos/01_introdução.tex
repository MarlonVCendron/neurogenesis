\chapter{Introdução}

Ao hipocampo atribui-se fundamental papel na codificação, consolidação e recordação de memórias
declarativas~\cite{eichenbaumHippocampus1999}. Para isso, o hipocampo recebe aferências de múltiplas áreas de associação
corticais, onde informações de diversas modalidades sensoriais são processadas. Essas informações convergem através da região
para-hipocampal, que compreende os córtices perirrinal, pós-rinal e, de principal importância, o córtex entorrinal (EC,
\textit{Entorhinal Cortex})~\cite{eichenbaumCortical2000}. Essa rica convergência de informações posiciona o hipocampo de forma
ideal para codificar associações entre eventos, consolidando seu papel central no aprendizado e na memória~\cite{henkeModel2010,
berdugo-vegaSharpening2023}.

Simplista e tradicionalmente, a estrutura do hipocampo tem sido descrita como sendo composta principalmente pelo circuito
tri-sináptico: onde a informação flui em um único sentido do EC ao DG, ao CA3 e ao CA1. O EC serve como principal entrada para o
giro denteado (DG, \textit{Dentate Gyrus}), através da via perfurante (PP, \textit{Perforant Path}), seus neurônios se conectam
principalmente com as células granulares (GC, \textit{Granule Cell}), as principais células excitatórias do DG, numa razão de
$1:10$; importantemente, o EC também envia aferências para CA3 e CA1, o que vai contra a noção simplista do circuito
tri-sináptico~\cite{basuCorticohippocampal2015}. Justamente por essa conectividade divergente do EC para o DG, acredita-se que o
DG seja responsável por uma função computacional chamada separação de padrões, através de sua atividade esparsa e da frequência de
disparos das GCs~\cite{hainmuellerDentate2020, kesnerMnemonic2006, yassaPattern2011}. A separação de padrões desambigua
informações similares, permitindo identificar e diferenciar experiências distintas mas extremamente similares. Para ilustrar essa
computação, pode-se considerar o seguinte exemplo excessivamente simplificado: experiências de um mesmo lugar, com as mesmas
pessoas e objetos presentes e em uma rotina similar são provavelmente codificadas no EC de forma muito similar, mas devido à
separação de padrões desempenhada pelo DG, pequenos detalhes como uma conversa específica ou uma notícia na televisão podem ser
suficientes para elicitar um padrão de atividade neural no DG muito distinto ao de experiências passadas, permitindo a codificação
de uma nova memória~\cite{eichenbaumHippocampus2004}.

Em contrapartida à projeção divergente do EC para o DG, a projeção do DG para o CA3 é altamente convergente. A região CA3 é
caracterizada por uma rede de conexões recorrentes, onde suas células piramidais (PCA3s) excitam tanto interneurônios quanto
outras PCA3s. Essa arquitetura forma uma rede auto-associativa que permite o armazenamento de padrões como assembleias neuronais:
grupos de neurônios que representam um mesmo padrão ao dispararem juntos, formados a partir da plasticidade neural que
potencializa as conexões de neurônios que disparam juntos. Uma vez que uma assembleia é formada pelo fortalecimento das sinapses
entre suas células, ela pode ser reativada posteriormente por um sinal de entrada parcial do padrão inicial, visto que ao ativar
uma subpopulação de células da assembleia, as demais também serão ativadas por conta da associação formada pela potenciação
sináptica. Esse processo é conhecido como completamento de padrões~\cite{kopsickFormation2024, leduigouRecurrent2014}. Através dos
colaterais de Schaffer, o CA3 projeta-se ao CA1, que atua como uma interface entre o hipocampo e o neocórtex, onde memórias são
consolidadas e recordadas~\cite{bartschCA12011}.

Além do seu papel fundamental na separação de padrões, o DG é uma das únicas áreas do encéfalo de mamíferos que apresenta
neurogênese adulta, a formação de novos neurônios a partir de células-tronco após o desenvolvimento, gerando células granulares
continuamente durante a vida~\cite{boldriniHuman2018, dumitruIdentification2025}. A neurogênese adulta em mamíferos ocorre apenas
em algumas outras áreas específicas, como no bulbo olfatório em roedores e possivelmente no estriado, amígdala e hipotálamo em
humanos~\cite{alonsoImpact2024, jurkowskiHippocampus2020}.

As GCs geradas pela neurogênese adulta levam algumas semanas para se desenvolverem completamente em roedores. Em 3 semanas, a
grande maioria das GCs imaturas (iGCs) morre, e, a partir de aproximadamente 8 semanas, elas se tornam praticamente
indistinguíveis das GCs maduras (mGCs)~\cite{denoth-lippunerFormation2021}. Ao longo da maturação, as iGCs sofrem diversas
alterações morfológicas e eletrofisiológicas, expandindo seus dendritos e formando conexões com o CA3. Aproximadamente entre 4 e 6
semanas ocorre um período crítico em que as iGCs, já parcialmente integradas ao circuito, apresentam uma maior excitabilidade e
plasticidade sináptica quando comparadas às mGCs~\cite{zhaoDistinct2006,denoth-lippunerFormation2021, aimoneRegulation2014},
oferecendo uma janela crítica para a plasticidade e a codificação de memória~\cite{berdugo-vegaSharpening2023}.

Dada a dificuldade de experimentação no DG e a complexidade da informação processada no hipocampo, modelos computacionais têm sido
inestimáveis para compreender o papel da neurogênese, que ainda não é bem entendida~\cite{aimoneComputational2016}. Modelos
computacionais variam em escala, realismo biológico, circuitos modelados e diversos outros fatores; portanto há uma gama de
diferentes resultados acerca da função da neurogênese, mas as principais teorias postulam que as iGCs estão envolvidas ou
diretamente com a codificação de memórias, ou como moduladoras do sistema~\cite{aimoneComputational2016,
berdugo-vegaSharpening2023}.


% TODO: parágrafos