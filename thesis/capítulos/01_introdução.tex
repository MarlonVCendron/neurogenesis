\chapter{Introdução}

Ao hipocampo atribui-se fundamental papel na codificação, consolidação e recordação de memórias declarativas~\cite{eichenbaumHippocampus1999}. O
hipocampo dos mamíferos recebe aferências de diversas áreas de associação do cérebro, onde informações de várias modalidades
sensoriais são combinadas, através do córtex para-hipocampal, que inclui, de principal importância, o córtex entorrinal (CE)
~\cite{eichenbaumCortical2000}. Com toda essa informação sensorial convergindo para o hipocampo, ele codifica associações de eventos
experienciados, servindo sua principal função de aprendizado e memória~\cite{berdugo-vegaSharpening2023, henkeModel2010}.

Simplista e tradicionalmente, a estrutura do hipocampo tem sido descrita como sendo composta principalmente pelo circuito
tri-sináptico (GD - CA3 - CA1). O CE serve como principal entrada para o giro denteado (GD), através da via perfurante (VP), seus
neurônios se conectam com as células granulares (CG) numa razão de 1:10; importantemente, o CE também envia aferências para CA3 e
CA1, o que vai contra a noção simplista de que o hipocampo é composto apenas pelo circuito tri-sináptico~\cite{basuCorticohippocampal2015}.
Justamente por essa conectividade divergente do CE para o GD, acredita-se que o GD seja responsável por uma função
computacional chamada separação de padrões, através de sua atividade esparsa e da frequência de disparos das CGs~\cite{hainmuellerDentate2020, kesnerMnemonic2006, yassaPattern2011}.
A separação de padrões desambigua informações similares, permitindo
identificar e diferenciar experiências distintas mas extremamente similares, por exemplo experiências de um mesmo lugar, com as
mesmas pessoas e objetos presentes e em uma rotina similar~\cite{eichenbaumHippocampus2004}.

A projeção do giro denteado (GD) para o CA3 é altamente convergente. As células piramidais do CA3 (PCA3) excitam interneurônios e
outras PCA3s, formando uma rede associativa com conexões recorrentes em uma ampla área, mas com baixa conectividade, permitindo a
aquisição e recordação de representações neurais através do completamento de padrões~\cite{leduigouRecurrent2014}. Através dos
colaterais de Schaffer, o CA3 projeta-se ao CA1, que atua como uma interface entre o hipocampo e o córtex, onde memórias são
consolidadas e recordadas~\cite{bartschCA12011}.

Além do seu papel fundamental na separação de padrões, o GD é uma das únicas áreas do encéfalo de mamíferos que apresenta
neurogênese adulta, a formação de novos neurônios a partir de células-tronco após o desenvolvimento, gerando células granulares
jovens continuamente durante a vida~\cite{boldriniHuman2018}. A neurogênese adulta em mamíferos ocorre apenas em algumas outras
áreas específicas, como no bulbo olfatório em roedores e possivelmente no estriado, amígdala e hipotálamo em
humanos~\cite{alonsoImpact2024, jurkowskiHippocampus2020}. CGs jovens (CGj) apresentam características eletrofisiológicas
distintas cerca de quatro a seis semanas após o nascimento, período em que formam conexões com o CA3~\cite{zhaoDistinct2006},
desenvolvem dendritos complexos e mostram maior plasticidade sináptica e hiperexcitabilidade comparadas às CGs adultas
(CGas)~\cite{aimoneRegulation2014}, oferecendo uma janela crítica para a plasticidade e a codificação de
memória~\cite{berdugo-vegaSharpening2023}.

Dada a dificuldade de experimentação no GD e a complexidade da informação processada no hipocampo, modelos computacionais têm sido
inestimáveis para compreender o papel da neurogênese, que ainda não é bem entendida~\cite{aimoneComputational2016}. Modelos computacionais variam
em escala, realismo biológico, circuitos modelados e diversos outros fatores; portanto há uma gama de diferentes resultados acerca
da função da neurogênese, mas as principais teorias postulam que as CGjs estão envolvidas ou diretamente com a codificação de
memórias, ou como moduladoras do sistema~\cite{aimoneComputational2016, berdugo-vegaSharpening2023}.

Este projeto de mestrado propõe uma abordagem inovadora ao empregar um modelo computacional do hipocampo para investigar a função da neurogênese adulta no GD e as consequências dela no CA3, com o objetivo de
completar lacunas no entendimento da função da neurogênese na codificação, armazenamento e recordação de memórias.

