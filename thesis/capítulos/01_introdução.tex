\chapter{Introdução}

Ao hipocampo atribui-se fundamental papel na codificação, consolidação e recordação de memórias
declarativas~\cite{eichenbaumHippocampus1999, liaoLearning2024}. Para isso, o hipocampo recebe aferências de múltiplas áreas de
associação corticais, onde informações de diversas modalidades sensoriais são processadas. Essas informações convergem através da
região para-hipocampal, que compreende os córtices perirrinal, pós-rinal e, de principal importância, o córtex entorrinal (EC,
\textit{Entorhinal Cortex})~\cite{eichenbaumCortical2000}. Essa rica convergência de informações posiciona o hipocampo de forma
ideal para codificar associações entre eventos, consolidando seu papel central no aprendizado e na memória~\cite{henkeModel2010,
berdugo-vegaSharpening2023}.

Simplista e tradicionalmente, a estrutura do hipocampo tem sido descrita como sendo composta principalmente pelo circuito
tri-sináptico: onde a informação flui em um único sentido do EC ao DG, ao CA3 e ao CA1. O EC serve como principal entrada para o
giro denteado (DG, \textit{Dentate Gyrus}), através da via perfurante (PP, \textit{Perforant Path}), cujos neurônios se conectam
principalmente com as células granulares (GC, \textit{Granule Cell}), as principais células excitatórias do DG, numa razão de
$1:10$; importantemente, o EC também envia aferências para CA3 e CA1, o que vai contra a noção simplista do circuito
tri-sináptico~\cite{basuCorticohippocampal2015}. Justamente por essa conectividade divergente do EC para o DG, acredita-se que o
DG seja responsável por uma função computacional chamada separação de padrões, através de sua atividade esparsa e da frequência de
disparos das GCs~\cite{hainmuellerDentate2020, kesnerMnemonic2006, yassaPattern2011}. A separação de padrões desambigua
informações similares, permitindo identificar e diferenciar experiências distintas mas extremamente similares, o que é fundamental
para evitar a interferência catastrófica entre memórias armazenadas no CA3~\cite{rollsMechanisms2013}. Para ilustrar essa
computação, pode-se considerar o seguinte exemplo excessivamente simplificado: experiências de um mesmo lugar, com as mesmas
pessoas e objetos presentes e em uma rotina similar são provavelmente codificadas no EC de forma muito similar, mas devido à
separação de padrões desempenhada pelo DG, pequenos detalhes como uma conversa específica ou uma notícia na televisão podem ser
suficientes para elicitar um padrão de atividade neural no DG muito distinto ao de experiências passadas, permitindo a codificação
de uma nova memória~\cite{eichenbaumHippocampus2004}.

Em contrapartida à projeção divergente do EC para o DG, a projeção do DG para o CA3 é altamente convergente. A região CA3 é
caracterizada por uma rede de conexões recorrentes, onde suas células piramidais (PCA3s) excitam tanto interneurônios quanto
outras PCA3s. Essa arquitetura forma uma rede auto-associativa que permite o armazenamento de padrões como assembleias neuronais:
grupos de neurônios que representam um mesmo padrão ao dispararem juntos, formados a partir da plasticidade neural dependente do
tempo de disparo (STDP) que potencializa as conexões de neurônios que disparam juntos~\cite{kopsickFormation2024}. Uma vez que uma
assembleia é formada pelo fortalecimento das sinapses entre suas células, ela pode ser reativada posteriormente por um sinal de
entrada parcial do padrão inicial, visto que ao ativar uma subpopulação de células da assembleia, as demais também serão ativadas
por conta da associação formada pela potenciação sináptica. Esse processo é conhecido como completamento de padrões, e modelos
computacionais recentes demonstram que essa arquitetura permite a recuperação robusta de memórias mesmo a partir de pistas
degradadas~\cite{kopsickFormation2024, leduigouRecurrent2014}. Através dos colaterais de Schaffer, o CA3 projeta-se ao CA1, que
atua como uma interface entre o hipocampo e o neocórtex, onde memórias são consolidadas e recordadas~\cite{bartschCA12011}.

Além do seu papel fundamental na separação de padrões, o DG é uma das únicas áreas do encéfalo de mamíferos que apresenta
neurogênese adulta, a formação de novos neurônios a partir de células-tronco após o desenvolvimento, gerando células granulares
continuamente durante a vida~\cite{boldriniHuman2018, dumitruIdentification2025}. A neurogênese adulta em mamíferos ocorre apenas
em algumas outras áreas específicas, como no bulbo olfatório em roedores e possivelmente no estriado, amígdala e hipotálamo em
humanos~\cite{alonsoImpact2024, jurkowskiHippocampus2020}.

As GCs geradas pela neurogênese adulta levam algumas semanas para se desenvolverem completamente em roedores. Em 3 semanas, a
grande maioria das GCs imaturas (iGCs) morre, e, a partir de aproximadamente 8 semanas, elas se tornam praticamente
indistinguíveis das GCs maduras (mGCs)~\cite{denoth-lippunerFormation2021}. Ao longo da maturação, as iGCs sofrem diversas
alterações morfológicas e eletrofisiológicas, expandindo seus dendritos e formando conexões com o CA3. Aproximadamente entre 4 e 6
semanas ocorre um período crítico em que as iGCs, já parcialmente integradas ao circuito, apresentam uma maior excitabilidade e
plasticidade sináptica quando comparadas às mGCs~\cite{zhaoDistinct2006,denoth-lippunerFormation2021, aimoneRegulation2014},
oferecendo uma janela crítica para a plasticidade e a codificação de memória~\cite{berdugo-vegaSharpening2023}.

De forma geral, a neurogênese adulta está implicada em todas as fases da memória, incluindo codificação, consolidação, recordação
e até mesmo o esquecimento~\cite{chavanMemory2025}. As iGCs desempenham um papel crucial na detecção de novidades e na separação
de padrões, o que permite a discriminação entre memórias semelhantes. Além disso, evidências recentes sugerem um papel
especializado para esses neurônios na consolidação da memória durante o sono REM~\cite{chavanMemory2025} e também no processo de
consolidação que generaliza as memórias, sendo essenciais nas mudanças estruturais de circuitos no CA3 através da ativação
promíscua de assembleias neuronais por sua alta excitabilidade~\cite{koSystems2025}.

Apesar de diversos estudos demonstrarem o papel da neurogênese adulta em diferentes funções do hipocampo~\cite{chavanMemory2025,
berdugo-vegaSharpening2023}, os mecanismos computacionais em nível de circuito, pelos quais as iGCs desempenham essas funções,
permanecem pouco compreendidos.

Essa questão central se desdobra em duas principais linhas teóricas sobre a função das iGCs~\cite{berdugo-vegaSharpening2023}. A
primeira hipótese é que as iGCs se integram diretamente aos engramas de memória a longo prazo. Nesse cenário, elas atuariam na
codificação de novas informações, na integração de padrões ou na prevenção de interferência catastrófica com memórias já
estabelecidas pelas mGCs. A segunda hipótese, que tem sido explorada mais recentemente, propõe que as iGCs funcionam como
elementos moduladores temporários do circuito. Nesse papel, elas poderiam tanto regular a atividade do DG, aumentando a
esparsidade da ativação ao inibir as mGCs, quanto modular a atividade do CA3~\cite{aimoneComputational2016,
berdugo-vegaSharpening2023}, ao invés de serem diretamente recrutadas na codificação das memórias.

Dada a dificuldade de experimentação no DG e a complexidade da informação processada no hipocampo, modelos computacionais têm sido
inestimáveis para compreender o papel da neurogênese~\cite{aimoneComputational2016}. Modelos computacionais variam em escala,
realismo biológico, circuitos modelados e diversos outros fatores; portanto há uma gama de diferentes resultados acerca da função
e dos mecanismos da neurogênese~\cite{aimoneComputational2016, berdugo-vegaSharpening2023}.

Em um modelo de condutância do DG~\cite{kimEffect2024}, a grande população de mGCs desempenhou uma ótima separação de padrões,
enquanto a minoria das iGCs serviu como integradora de padrões. A presença das iGCs deteriorou a capacidade de separação de
padrões do DG em comparação com um DG composto apenas por mGCs, porém os autores postulam que essa codificação mista de integração
e separação de padrões poderia aumentar a capacidade de armazenamento e recuperação de memórias, hipótese esta que não puderam
testar pois somente o DG foi modelado, e não outras áreas como CA1 e CA3. Curiosamente, esses resultados vão contra estudos
experimentais que demonstram que um aumento na neurogênese aumenta a separação de padrões~\cite{sahayIncreasing2011}, o que pode
ser um indicativo de que os modelos computacionais do hipocampo ainda não conseguem capturar por completo as computações
desempenhadas pelo mesmo.

Em contraste, o trabalho de~\citeonline{yangDynamic2025} utilizou um modelo do DG para investigar o impacto dinâmico da
neurogênese na separação de padrões. Seus resultados indicam que o efeito geral das iGCs sobre as GCs é inibitório, o que aumenta
a esparsidade da rede e, consequentemente, melhora a separação de padrões, fornecendo evidências computacionais para a hipótese de
``codificação indireta''. Os autores também investigaram como as características das iGCs durante a maturação, como a diminuição
da atividade e o aumento da plasticidade sináptica, afetam dinamicamente a separação de padrões. O estudo também sugere que a
presença de iGCs aumenta a adaptabilidade da rede do DG a diferentes frequências de entrada, melhorando a robustez da separação de
padrões.

O modelo de~\citeonline{aimoneComputational2009} foi feito em um nível de realismo biológico intermediário sem usar um modelo de
condutância, o que permitiu investigar as iGCs durante toda a sua maturação através do tempo. Ao modelar o tempo de maturação
dessa população de iGCs, o modelo mostrou que elas codificaram preferencialmente informações recebidas de quando eram jovens,
portanto apresentando o papel de integradoras temporais. Esse modelo foi corroborado por estudos
experimentais~\cite{berdugo-vegaSharpening2023}, mas não analisou as implicações para outras áreas como CA1 e CA3, modelando
apenas o DG.

No trabalho de~\citeonline{kassabPattern2018a}, os autores criaram um modelo do circuito DG-CA3 com uma nova proposta de
diferentes circuitos paralelos no hipocampo com funções distintas. Não foi o objetivo deste trabalho estudar ou modelar a
neurogênese, mas os autores sugerem a possibilidade de que ela possa ser a responsável por um dos circuitos postulados no estudo.

Este projeto de mestrado propõe o desenvolvimento de um modelo computacional do circuito DG-CA3 para investigar a função da
neurogênese adulta. Diferentemente de muitos trabalhos que se concentram apenas no DG ou em populações estáticas de células
imaturas, este estudo adotará uma abordagem mais holística e, portanto, inovadora. O modelo simulará a maturação temporal das
células iGCs em mGCs, permitindo uma análise dinâmica de como a contínua integração de novos neurônios afeta o circuito ao longo
do tempo. O objetivo principal é analisar os impactos da neurogênese não apenas na separação de padrões do DG, mas também nas
funções de autoassociação e completamento de padrões da área CA3. Serão investigados os efeitos de diferentes níveis de
integração das iGCs, sua maior excitabilidade e como as retroprojeções do CA3 para o DG, um mecanismo de feedback frequentemente
negligenciado em modelos~\cite{myersPattern2011}, influenciam a dinâmica da rede. Com isso, busca-se entender como as iGCs,
durante sua fase imatura e após a maturação, modulam a formação e a reativação de memórias no CA3.
