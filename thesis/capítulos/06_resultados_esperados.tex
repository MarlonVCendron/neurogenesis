\chapter{Resultados}

\section{Separação de padrões}

% falo aqui ou em métodos:
%  30 trials, etc

\section{Resultados esperados}

Espera-se que a presença de iGCs prejudique o completamento de padrões no CA3, devido a uma maior ativação de assembleias
neuronais não relacionadas ao padrão de entrada, um efeito consistente com resultados experimentais que sugerem uma ativação
promíscua de assembleias por parte das iGCs devido à sua alta excitabilidade~\cite{koSystems2025}. Adicionalmente, na simulação da
maturação temporal, espera-se que as iGCs passem a codificar os padrões no tempo, integrando informações que foram apresentadas
enquanto eram jovens, corroborando seu papel como integradoras temporais, como proposto por~\cite{aimoneComputational2009}.
Teoriza-se, portanto, que as mGCs teriam um papel predominante na separação de padrões específicos e distintos, enquanto as iGCs,
ao longo de sua maturação, se especializariam na separação de padrões próximos no tempo, ou seja, aqueles que codificaram durante
sua fase imatura.
