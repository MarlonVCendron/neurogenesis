\chapter{Hipóteses}

Com base nos resultados preliminares e na literatura, a hipótese central deste trabalho é que o modelo computacional revelará uma
dicotomia funcional entre as populações de células granulares. Espera-se que as mGCs se comportem como separadoras de padrões,
aumentando a dissimilaridade entre as representações neurais de estímulos distintos. Em contrapartida, prevê-se que as iGCs, em
virtude de sua maior excitabilidade intrínseca, funcionem como integradoras de padrões, com sua ativação promíscua diminuindo a
distinção entre padrões de saída e, consequentemente, degradando a capacidade de separação de padrões do DG como um todo.

Adicionalmente, hipotetiza-se que essa ativação indiscriminada das iGCs impactará negativamente as funções de memória do CA3,
prejudicando a fidelidade do completamento de padrões ao coativar assembleias neuronais não relacionadas ao estímulo. Embora a
atividade das iGCs possa suprimir a das mGCs através de circuitos inibitórios compartilhados, espera-se que essa supressão aumente
marginalmente a esparsidade das mGCs, melhorando levemente sua capacidade de separação. Por fim, o modelo testará a hipótese de
que as iGCs atuam como integradoras temporais, com sua maturação especializando-as para responder a informações codificadas
durante sua fase imatura, estabelecendo um mecanismo para a separação de memórias no tempo.

