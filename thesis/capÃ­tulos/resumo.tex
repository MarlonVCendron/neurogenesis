\setlength{\absparsep}{18pt} % ajusta o espaçamento dos parágrafos do resumo

\begin{resumo}

A neurogênese adulta no giro denteado (DG) do hipocampo está implicada em diversas funções da memória, desde a codificação e
consolidação até o esquecimento. Apesar das evidências experimentais sobre seus efeitos macro, os mecanismos computacionais pelos
quais as células granulares imaturas (iGCs) geradas pela neurogênese adulta modulam o circuito hipocampal permanecem pouco
compreendidos. Estudos sobre o tema frequentemente se concentram nos efeitos locais no DG, negligenciando o impacto em áreas
subsequentes como o Cornu Ammonis 3 (CA3), e tratam as iGCs como uma população estática, uma simplificação que não captura a
dinâmica de sua maturação. Este projeto propõe o desenvolvimento de um modelo computacional do circuito DG-CA3 com alto grau de
fidelidade biológica para investigar essas questões. O trabalho analisará como as iGCs, durante sua fase de maior excitabilidade e
após sua maturação em células maduras (mGCs), influenciam não apenas a separação de padrões do DG, mas também as funções de
auto-associação e completamento de padrões do CA3. O modelo simulará a maturação temporal das iGCs e a adição contínua de novos
neurônios, além de incorporar as retroprojeções do CA3 para o DG, buscando assim uma compreensão mais aprofundada dos mecanismos
pelos quais a neurogênese modula a dinâmica do circuito hipocampal.

\noindent\textbf{Palavras-chave}: Neurogênese adulta, hipocampo, modelo computacional, plasticidade, giro denteado, CA3.
\end{resumo}

% \begin{resumo}[Abstract]
% \begin{otherlanguage*}{english}
% Abstract
   
% \noindent \textbf{Keywords}: Word1. Word2. Word3. Word4. Word5.
% \end{otherlanguage*}
% \end{resumo}



